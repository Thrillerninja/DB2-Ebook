\chapter{Type of Database \& History}

Couchbase is a distributed NoSQL database that combines document database and key-value store capabilities. Unlike traditional single-model NoSQL solutions, Couchbase offers a hybrid approach that makes it versatile for different use cases.

\section{Understanding Couchbase in the NoSQL Landscape}

NoSQL databases emerged to address limitations of relational systems when dealing with large data volumes and varied structures. The main types include:

\begin{itemize}
  \item Document databases: Store semi-structured data as documents (usually JSON)
  \item Key-value stores: Simple repositories storing values indexed by keys
  \item Column-family stores: Store data in column families instead of tables
  \item Graph databases: Specialize in managing highly connected data
\end{itemize}

Couchbase is primarily a distributed document database that also offers key-value capabilities, making it a multi-model solution.

\section{Key Characteristics and Features}

Couchbase offers a robust set of features that position it uniquely in the database market:

\subsection{Document Model}
Couchbase stores data as JSON documents, which offer several advantages:
\begin{itemize}
  \item Schema flexibility with no enforced document structure
  \item Support for nested attributes of arbitrary depth
  \item Ability to store arrays and embedded objects
  \item Document sizes up to 20MB (default limit)
  \item Metadata stored alongside documents (cas values, expiration)
  \item Document-level atomicity for write operations
\end{itemize}

\subsection{Key-Value Operations}
At its core, Couchbase provides high-performance key-value capabilities:
\begin{itemize}
  \item Sub-millisecond read/write operations
  \item Direct document access via unique keys
  \item Support for basic CRUD operations (create, read, update, delete)
  \item Optimistic concurrency control with CAS (Compare-And-Swap) values
  \item Document expiration with TTL (Time-To-Live) settings
  \item Atomic counter operations
\end{itemize}

\subsection{Memory-First Architecture}
Couchbase's performance stems from its memory-centric design:
\begin{itemize}
  \item Built-in caching layer derived from Memcached technology
  \item Configurable memory quotas per service
  \item Managed cache eviction based on ejection policies
  \item Background persistence to disk
  \item Working set management for optimizing memory use
  \item Direct memory access capabilities
\end{itemize}

\subsection{N1QL Query Language}
N1QL (pronounced "nickel") extends SQL for JSON data:
\begin{itemize}
  \item SQL-like syntax for querying JSON documents
  \item Support for JOINs between multiple document collections
  \item Filtering on nested attributes and array elements
  \item Aggregation functions (COUNT, SUM, AVG, etc.)
  \item Subqueries and correlated subqueries
  \item Window functions for advanced analytics
\end{itemize}

\subsection{Indexing Capabilities}
Couchbase provides several indexing options to optimize queries:
\begin{itemize}
  \item Primary indexes for collection scans
  \item Secondary indexes for specific fields
  \item Composite indexes for multi-field queries
  \item Partial indexes with WHERE clauses
  \item Array indexes for querying array elements
  \item Adaptive indexes that learn from query patterns
\end{itemize}

\subsection{Clustering and Distribution}
Couchbase is built from the ground up as a distributed system:
\begin{itemize}
  \item Automatic sharding of data across cluster nodes
  \item Data replication with configurable replica counts
  \item Rack/zone awareness for resilience
  \item Auto-failover for node failures
  \item Online rebalancing without downtime
  \item Cross-datacenter replication (XDCR)
\end{itemize}

\subsection{Services Architecture}
Couchbase uses a multi-service architecture allowing independent scaling:
\begin{itemize}
  \item Data Service: Stores and serves documents
  \item Query Service: Processes N1QL queries
  \item Index Service: Manages indexes
  \item Search Service: Provides full-text search capabilities
  \item Analytics Service: Handles analytical queries
  \item Eventing Service: Processes data change events
\end{itemize}

\subsection{Security Features}
Enterprise-grade security includes:
\begin{itemize}
  \item Role-Based Access Control (RBAC)
  \item TLS encryption for data in transit
  \item Encrypted administrative credentials
  \item Auditing of database operations
  \item LDAP integration for authentication
  \item Field-level encryption for sensitive data
\end{itemize}

\section{Brief History}

Couchbase was formed in 2010 through the merger of Membase (a key-value store with Memcached compatibility) and CouchDB technology (document database capabilities). Key milestones include:

\begin{itemize}
  \item 2011: Release of Couchbase Server 1.0
  \item 2015: Introduction of N1QL, a SQL-like query language for JSON
  \item 2017: Launch of Couchbase Mobile for edge computing
  \item 2021: Couchbase goes public with IPO
\end{itemize}

The introduction of N1QL in 2015 was particularly significant, as it bridged the gap between NoSQL flexibility and SQL familiarity.

\section{Use Case Examples}

Couchbase's feature set makes it particularly suited for specific use cases:

\subsection{User Profile Management}
Storing user profiles benefits from:
\begin{itemize}
  \item Document model for varying user attributes
  \item Key-value access for fast profile retrieval
  \item Expiration features for session management
\end{itemize}

\subsection{Product Catalogs}
E-commerce catalogs leverage:
\begin{itemize}
  \item Flexible schema for diverse product attributes
  \item N1QL for complex product searches
  \item Full-text search for product discovery
\end{itemize}

\subsection{Gaming Applications}
Online gaming uses:
\begin{itemize}
  \item Low-latency key-value operations for game state
  \item Document model for player profiles and game items
  \item Horizontal scaling for growing player bases
\end{itemize}

\subsection{Internet of Things (IoT)}
IoT deployments benefit from:
\begin{itemize}
  \item Mobile synchronization for edge devices
  \item Time-series capabilities for sensor data
  \item Flexible schema for diverse device data
\end{itemize}
