\chapter{Type of Database \& History}

Couchbase is a distributed NoSQL database that combines document database and key-value store capabilities. Unlike traditional single-model NoSQL solutions, Couchbase offers a hybrid approach that makes it versatile for different use cases.

% \section{Understanding Couchbase in the NoSQL Landscape}

% NoSQL databases emerged to address limitations of relational systems when dealing with large data volumes and varied structures. The main types include:

% \begin{itemize}
%   \item Document databases: Store semi-structured data as documents (usually JSON)
%   \item Key-value stores: Simple repositories storing values indexed by keys
%   \item Column-family stores: Store data in column families instead of tables
%   \item Graph databases: Specialize in managing highly connected data
% \end{itemize}

% Couchbase is primarily a distributed document database that also offers key-value capabilities, making it a multi-model solution.

\section{Key Characteristics and Features}

Couchbase distinguishes itself through several core capabilities that combine to create a versatile database platform:

\subsection{Data Model and Storage}
\begin{itemize}
  \item \textbf{JSON Document Storage}: Schema-flexible documents supporting nested attributes, arrays, and objects (up to 20MB per document) (\cite{couchabse_data})
  \item \textbf{Key-Value Operations}: Sub-millisecond CRUD operations with direct key access, optimistic concurrency (CAS), and TTL expiration (\cite{couchbaseSubMS})
  \item \textbf{Memory-First Architecture}: Built-in caching layer with configurable memory quotas and background disk persistence (\cite[Buckets, Memory, and Storage]{CouchbaseArchitecture2025})
\end{itemize}

\subsection{Query and Indexing}
\begin{itemize}
  \item \textbf{SQL++/N1QL Query Language}: SQL-like syntax for JSON with support for JOINs, nested attributes, arrays, aggregations, and subqueries (\cite[SQL++ versus SQL]{couchabse_data})
  \item \textbf{Comprehensive Indexing}: Primary, secondary, composite, partial, and array indexes to optimize query performance (\cite[Indexes]{couchabse_data})
\end{itemize}

\subsection{Distribution and Scalability}
\begin{itemize}
  \item \textbf{Distributed Architecture}: Automatic sharding, configurable replication, auto-failover, and online rebalancing (\cite{couchbase_services})
  \item \textbf{Multi-Service Design}: Independent scaling of data, query, index, search, analytics, and eventing services (\cite{couchbase_services})
  \item \textbf{Cross-Datacenter Replication}: Built-in XDCR for geographic distribution (\cite{couchabse_xdcr})
\end{itemize}

\subsection{Security and Enterprise Features}
\begin{itemize}
  \item \textbf{Access Control}: Role-Based Access Control (RBAC), LDAP integration, and operation auditing
  \item \textbf{Encryption}: TLS for data in transit and field-level encryption for sensitive data
\end{itemize}
(\cite{couchbase_security})


\section{Brief History}

Couchbase was formed in 2010 through the merger of Membase (a key-value store with Memcached compatibility) and CouchDB technology (document database capabilities). Key milestones include:

\begin{itemize}
  \item 2011: Release of Couchbase Server 1.0
  \item 2015: Introduction of N1QL, a SQL-like query language for JSON
  \item 2017: Launch of Couchbase Mobile for edge computing
  \item 2021: Couchbase goes public with IPO
\end{itemize}

The introduction of N1QL in 2015 was particularly significant, as it bridged the gap between NoSQL flexibility and SQL familiarity. (\cite{cbhistory})

\section{Use Case Examples}

Couchbase excels in several common application scenarios:

\textbf{User Profile Management:} Document model for dynamic attributes, fast key-value access, and session expiration.

\textbf{Product Catalogs:} Flexible schema for product attributes, N1QL for advanced queries, and full-text search.

\textbf{Gaming Applications:} Low-latency key-value operations, document model for player/game data, and horizontal scaling.

\textbf{Internet of Things:} Mobile sync for edge devices, time-series support for sensor data, and schema flexibility for device diversity.