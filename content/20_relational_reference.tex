\chapter{Which aspects of relational DBs are improved?}

Couchbase addresses several fundamental limitations of relational databases. This chapter examines these improvements through direct comparisons with traditional relational systems.

\section{Schema Flexibility vs. Schema Rigidity}

\subsection{The Relational Challenge}
In relational databases, schemas must be predefined before data insertion:

\begin{itemize}
  \item Schema modifications require ALTER TABLE operations
  \item Changes often require migrations and downtime
  \item Application code frequently needs updates for schema changes
  \item Development agility is restricted by schema rigidity
\end{itemize}

\subsection{Couchbase's Solution}
Couchbase uses a document model that fundamentally rethinks this approach:

\begin{itemize}
  \item Documents can have different structures within the same collection
  \item Fields can be added or removed per document
  \item Applications can evolve without database migrations
  \item Different versions of document structures can coexist
\end{itemize}

For example, two products can have different structures without requiring schema changes:

\begin{verbatim}
// Product with basic attributes
{
  "id": "product123",
  "type": "product",
  "name": "Laptop",
  "price": 999.99
}

// Product with additional attributes
{
  "id": "product456",
  "type": "product",
  "name": "Smartphone",
  "price": 699.99,
  "colors": ["black", "silver", "gold"],
  "specs": {
    "cpu": "Snapdragon 8 Gen 2",
    "memory": "8GB"
  }
}
\end{verbatim}

\section{Scalability Approaches}

\subsection{The Relational Challenge}
Relational databases were designed for vertical scaling:
\begin{itemize}
  \item Scaling usually means buying bigger servers
  \item Reaching hardware limits becomes inevitable
  \item Sharding requires complex application logic
  \item Joins become problematic across shards
\end{itemize}

\subsection{Couchbase's Solution}
Couchbase was designed from the beginning for distributed operation:
\begin{itemize}
  \item Horizontal scaling by adding commodity servers
  \item Built-in auto-sharding across the cluster
  \item Native replication for high availability
  \item Multi-dimensional scaling (separate query, index, and data services)
\end{itemize}

\section{Query Language Evolution}

\subsection{The Relational Challenge}
SQL has limitations when dealing with modern application data:
\begin{itemize}
  \item Not naturally suited for nested structures
  \item Complex structures must be split across tables
  \item Joins can become performance bottlenecks
  \item Mismatch with object-oriented application code
\end{itemize}

\subsection{Couchbase's Solution}
N1QL (SQL for JSON) provides:
\begin{itemize}
  \item SQL-like syntax familiar to database developers
  \item Native operations on nested JSON structures
  \item Direct support for arrays and objects
  \item Less reliance on joins due to document embedding
\end{itemize}

Example N1QL query:
\begin{verbatim}
SELECT p.name, p.price, 
       ARRAY_LENGTH(p.colors) AS color_options,
       p.specs.cpu
FROM products AS p
WHERE p.price < 1000
  AND "black" IN p.colors
  AND p.specs.memory = "8GB";
\end{verbatim}

\section{Relationship to Codd's Rules}

Couchbase intentionally diverges from some of Codd's relational principles:

\begin{itemize}
  \item \textbf{Information Rule}: Uses flexible JSON documents instead of tabular rows
  \item \textbf{Guaranteed Access}: Accesses data via document IDs and paths rather than tables and columns
  \item \textbf{Systematic Treatment of Nulls}: Missing attributes simply don't exist in documents
  \item \textbf{Logical Data Independence}: Achieves this differently through schema flexibility
\end{itemize}

These differences represent deliberate design choices for modern application development rather than deficiencies.
