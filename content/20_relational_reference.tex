\chapter{Which aspects of relational DBs are improved?}

Couchbase addresses three fundamental limitations of relational databases: schema rigidity, scaling challenges, and query limitations for complex data structures.

\section{Schema Flexibility}

\textbf{Relational limitation:} Relational databases require predefined schemas where all data must conform to a fixed structure. Schema changes typically involve ALTER TABLE operations that often cause downtime, require data migrations, and necessitate application code updates.

\textbf{Couchbase solution:} Couchbase's document model eliminates rigid schemas by allowing each document to have its own structure, even within the same collection. This enables:
\begin{itemize}
  \item Adding or removing fields without database migrations
  \item Evolving applications without downtime
  \item Supporting multiple document versions simultaneously
\end{itemize}

For example, while a simple product might contain just basic attributes (id, name, price), another can include nested structures like arrays of colors and specification objects—all without schema changes.

\section{Scalability Improvements}

\textbf{Relational limitation:} Traditional relational databases were designed for vertical scaling (bigger servers) rather than horizontal scaling. As data grows, this approach eventually hits hardware limits. Manual sharding requires complex application logic, and joins become problematic across distributed data.

\textbf{Couchbase solution:} Couchbase is architected for distributed operations from the ground up, featuring automatic sharding, built-in replication, seamless cluster expansion, and multi-dimensional scaling that allows separate scaling of query, index, and data services.

\section{Advanced Query Capabilities}

\textbf{Relational limitation:} SQL struggles with hierarchical data structures common in modern applications. Complex nested structures must be split across multiple tables, requiring joins that become performance bottlenecks at scale. This creates a mismatch with object-oriented application code.

\textbf{Couchbase solution:} N1QL (SQL for JSON) combines SQL familiarity with direct operations on nested JSON structures:

\begin{verbatim}
SELECT product.name, product.specs.cpu
FROM products AS product
WHERE "black" IN product.colors;
\end{verbatim}

This query directly accesses nested fields and array elements without complex joins or subqueries.

\section{Relationship to Codd's Rules}

Couchbase intentionally diverges from relational principles to address modern application needs. While relational databases store data in rigid tables (Codd's Information Rule), Couchbase uses flexible JSON documents. This design choice prioritizes adaptability and development speed over traditional relational constraints.
